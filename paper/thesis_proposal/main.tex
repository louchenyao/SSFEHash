\documentclass{article}
\usepackage[utf8]{inputenc}

\title{Thesis Proposal: Supersuper Fast and Efficient Hashing}
\author{Chenyao Lou}

\usepackage{natbib}
\usepackage{graphicx}

\begin{document}

\maketitle

\section{Introduction}

% The hash table is important: Speed, and space efficiency

The Hash table is a very crucial important component for systems. For example, the load balancer uses the hash table to decide the associated backend server for specified connections. Or it can store the Forwarding Information Base (FIB) for network applications [Dong, CuckooSwitch] like SDN. The lookup of FIB is the biggest bottleneck either for software packets forwarding or hardware packets forwarding. Thus, developers desire a high-performance hash table for such tasks.

Besides, the space efficiency is also an important metric for hash tables. It has two benefits. First, the small query structure can fit into the CPU cache, which is ten times faster than the main memory (DRAM). Thus the lookup performance an be accelerated further with the same hardware. Second, a space efficient hash table storing more items benefits memory-constrained tasks.

% variations of the hash table

Therefore, we propose a new space efficient hash table SSFE (Supersuper Fast Hashing) design with the high lookup performance. The basic functionality of a hash table is that, given a set of tuples $\{(k_0, v_0), (k_1, v_1), ..., (k_{n-1}, v_{n-1})\}$, the hash table can give $v_i$ for the question $q(k_i)$. There are variations of this definition: For a $k \not\in \{k_0, ..., k_{n-1}\}$, (1) the hash table returns $\bot$ (key is not existing), or (2) returns a random $v \in \mathcal{V}$. SSFE is designed for the second definition, i.e. it returns a random value when the query key has never been seen before.

% basic idea

\section{Related Work}

\section{Design}

\section{Conclusion}


\bibliographystyle{plain}
\bibliography{references}
\end{document}
